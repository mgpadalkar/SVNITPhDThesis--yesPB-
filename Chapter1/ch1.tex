\chapter{Introduction}
\label{ch:intro}

Your first chapter. Go on and place some figures as given below.
\begin{figure}[!ht]
 	\centering
 	\subfigure[something here]{
 	\fbox
 	{\includegraphics[width = 0.45\textwidth]{./Figures/SVNITlogo}}
 		\label{fig:sf1}
 	}
 	\subfigure[without border]{
 	 {\includegraphics[width=0.45\textwidth]{./Figures/SVNITlogo}}
 		\label{fig:sf2}
 	}
 	\caption[Use this if required, which goes in `list of figures']{The caption is here. I can refer to the subfigures \subref{fig:sf1} and \subref{fig:sf2}. In case the subcaption-number i.e. \subref{fig:sf1} is not to be displayed above, then do not use [ ] in the subfigure command.}
 	\label{fig:f1}
\end{figure}
 
I can also refer to the sub-equations using \ref{fig:sf1} and \ref{fig:sf2} in figure \ref{fig:f1}. Nothing to say about the references. You could refer this way \cite{areport,Hill,MATLAB} or this way \citet{Akyildiz}. The citet command is possible due to the ``natbib'' package and ``IEEETranN.bst'' file.

\section{First Section}
Lets move ahead with tables.

\begin{table}[ht!]
	 \caption{My first table} % title of Table
  \label{table:t1} % is used to refer this table in the text
  \centering % used for centering table
 \begin{tabular}{l l l l l} % centered columns (4 columns)
 \hline\hline %inserts double horizontal lines
Technique   &H/W  & Distance  & Limitations \\ [0.5ex] % inserts table heading
 \hline % inserts single horizontal line
RSSI     & No              & Few Meters  & Noise, Interference in range\\
ToA      & Yes             & Few Cms     & Nodes synchronization\\
TDoA     & Ultrasound Txr  & Few Meters  & Maximum distance of work\\
AoA      & Set of receivers & few degrees & Work on small sensor nodes\\[0.5ex] % [1ex] adds vertical space
 \hline %inserts single line
 \end{tabular}
\end{table}
 
I can always refer this table \ref{table:t1} using its label. We can include the equations as well. Both environments viz. begin\{equation\} -- end\{equation\} and begin\{eqnarray\} -- end\{eqnarray\} are available. I personally prefer the later one. An example is given below in equation \ref{eq:e1}.

\begin{eqnarray}
x(t) = \left\{
	\begin{array}{ll}
		0, & \mbox{if}\ t<0, \\
		1, & \mbox{otherwise.}
	\end{array}
\right.
\label{eq:e1}
\end{eqnarray}

That's all from me. You may explore as much as you want.